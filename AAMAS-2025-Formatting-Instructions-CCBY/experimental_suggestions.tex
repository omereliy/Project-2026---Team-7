%%%%%%%%%%%%%%%%%%%%%%%%%%%%%%%%%%%%%%%%%%%%%%%%%%%%%%%%%%%%%%%%%%%%%%%%
% EXPERIMENTAL APPROACH SUGGESTIONS
% This file contains recommended changes to main.tex methodology section
% Each suggestion includes line references for where edits should be made
%%%%%%%%%%%%%%%%%%%%%%%%%%%%%%%%%%%%%%%%%%%%%%%%%%%%%%%%%%%%%%%%%%%%%%%%

\documentclass{article}
\usepackage{enumitem}
\usepackage{xcolor}
\usepackage{hyperref}

\title{Suggested Experimental Approach Revisions\\for NCF vs Focal Loss NCF Comparison}
\date{\today}

\begin{document}
\maketitle

\section{Summary of Issues}

The current experimental design has several methodological concerns that may undermine the validity of conclusions about Focal Loss effectiveness. This document proposes concrete changes.

%-----------------------------------------------------------------------
\section{Issue 1: Hypothesis Framework is Incomplete}
\textbf{Location:} main.tex lines 421--428 (Section 5.1 Research Hypotheses)

\subsection*{Current Problem}
Only H1 is defined, but H7 is referenced in line 613 (negative sampling ablation). Multiple research questions lack formal hypotheses.

\subsection*{Recommended Change}
Replace the current hypothesis section with:

\begin{verbatim}
\begin{description}
    \item[H1 (Focal Loss improves NeuMF):] NeuMF trained with
    Focal Loss achieves statistically significantly higher
    Hit Rate@10 and NDCG@10 compared to NeuMF trained with
    Binary Cross-Entropy loss, with a minimum relative
    improvement of 1\%.

    \item[H2 (Robustness to sampling ratio):] The performance
    gap between NeuMF-FL and NeuMF-BCE increases as the
    negative sampling ratio increases from 1:4 to 1:50.

    \item[H3 (Focusing effect is necessary):] NeuMF with
    $\gamma > 0$ (Focal Loss) outperforms NeuMF with
    $\gamma = 0$ (alpha-balanced BCE) when $\alpha$ is
    held constant, demonstrating that the focusing mechanism
    provides benefit beyond class reweighting.
\end{description}
\end{verbatim}

%-----------------------------------------------------------------------
\section{Issue 2: Training Imbalance is Artificially Reduced}
\textbf{Location:} main.tex line 455 (Section 5.2 Data Preprocessing)

\subsection*{Current Problem}
4:1 negative sampling reduces natural imbalance from 15--21:1 to 4:1, which is not extreme enough to demonstrate Focal Loss's advantage.

\subsection*{Recommended Change}
Restructure experiments with higher sampling ratios as primary:

\begin{verbatim}
\textbf{Primary Experiment Configuration:}
\begin{itemize}
    \item Training negative sampling ratio: 1:10 (moderate)
          and 1:20 (closer to natural imbalance)
    \item Evaluation: rank against 99 randomly sampled negatives
          (standard) PLUS full-ranking on a subset of users
\end{itemize}

\textbf{Ablation:} Test 1:4 (low) and 1:50 (high) as sensitivity
analysis to understand performance across the imbalance spectrum.
\end{verbatim}

%-----------------------------------------------------------------------
\section{Issue 3: Missing Control for Alpha-Balanced BCE}
\textbf{Location:} main.tex lines 498--515 (Table 3) and lines 602--603 (Ablation design)

\subsection*{Current Problem}
Focal Loss introduces two hyperparameters ($\gamma$, $\alpha$). Comparing against vanilla BCE conflates the focusing effect with class reweighting.

\subsection*{Recommended Change}
Add alpha-balanced BCE to baseline models table:

\begin{verbatim}
% Add to Table 3 (line ~515):
$\alpha$-BCE & Neural & Dot-product embeddings & $\alpha$-BCE \\

% Add to Table 5 (line ~555):
% New row in hyperparameter search:
$\alpha$ (for $\alpha$-BCE) & \{0.25, 0.5, 0.75\} & Class weight baseline \\
\end{verbatim}

Update ablation table (around line 630):
\begin{verbatim}
\begin{tabular}{lcccc}
\hline
\textbf{Study} & \textbf{BCE} & $\alpha$\textbf{-BCE} & \textbf{Focal Loss} \\
\hline
Baseline    & $\gamma=0, \alpha=1$  & $\gamma=0, \alpha$ tuned & $\gamma, \alpha$ tuned \\
$\gamma$ effect & -- & $\gamma=0$ & $\gamma \in \{0.5, 1, 2, 3\}$ \\
\hline
\end{tabular}
\end{verbatim}

%-----------------------------------------------------------------------
\section{Issue 4: Missing Training Dynamics Analysis}
\textbf{Location:} Insert after line 598 (before ablation studies list)

\subsection*{Current Problem}
The paper claims Focal Loss shifts gradient mass to hard examples but provides no direct evidence of this mechanism.

\subsection*{Recommended Addition}
Add new subsection or integrate into ablation:

\begin{verbatim}
\subsection{Training Dynamics Analysis}
\label{sec:training-dynamics}

To validate the mechanism underlying Focal Loss, we track the
following metrics during training:

\begin{enumerate}
    \item \textbf{Loss Contribution by Confidence Bin:}
    At each epoch, partition training samples into bins based on
    model confidence $p_t \in [0,0.2), [0.2,0.4), \ldots, [0.8,1.0]$.
    Compute the fraction of total loss contributed by each bin
    for BCE vs. Focal Loss.

    \item \textbf{Gradient Magnitude Analysis:}
    Track the L2 norm of gradients from easy samples
    ($p_t > 0.8$) vs. hard samples ($p_t < 0.5$) across epochs.

    \item \textbf{Convergence Speed on Hard Examples:}
    For a held-out set of ``hard'' user-item pairs (defined as
    pairs where early-epoch predictions are near 0.5), measure
    how quickly each loss function learns to classify them correctly.
\end{enumerate}

Expected outcome: Focal Loss should show (1) lower loss contribution
from high-confidence bins, (2) relatively higher gradient magnitude
from hard samples, and (3) faster convergence on hard examples
compared to BCE.
\end{verbatim}

%-----------------------------------------------------------------------
\section{Issue 5: Alpha-Sampling Interaction Effect}
\textbf{Location:} main.tex line 554 (Table 5)

\subsection*{Current Problem}
When $\alpha=0.5$ and negative sampling is 4:1, the effective class weights cancel out (positive weight 0.5 vs. negative weight 0.5, but 4x more negatives = 2:1 effective negative dominance, which is actually less than natural). This was observed empirically.

\subsection*{Recommended Change}
Add explicit analysis of interaction:

\begin{verbatim}
% Add to Section 5.5 (Ablation Study Design):
\item \textbf{Alpha-Sampling Interaction Analysis:}
To understand the interaction between class balancing and
negative sampling, we compute the \emph{effective class weight ratio}:
\begin{equation}
\text{Effective Ratio} = \frac{(1-\alpha) \times \text{neg\_ratio}}{\alpha}
\end{equation}
For example, with $\alpha=0.5$ and 4:1 sampling, effective ratio
$= (0.5 \times 4) / 0.5 = 4$, meaning negatives receive 4x total
weight despite balanced $\alpha$.

We test the following configurations to isolate effects:
\begin{center}
\begin{tabular}{ccc}
\hline
Neg Ratio & $\alpha$ & Effective Ratio \\
\hline
1:4 & 0.5 & 4:1 \\
1:4 & 0.8 & 1:1 (balanced) \\
1:10 & 0.5 & 10:1 \\
1:10 & 0.9 & $\approx$1:1 (balanced) \\
\hline
\end{tabular}
\end{center}
\end{verbatim}

%-----------------------------------------------------------------------
\section{Issue 6: Evaluation May Miss Hard-Case Advantage}
\textbf{Location:} main.tex line 580 (Section 5.4 Evaluation Protocol)

\subsection*{Current Problem}
Ranking against 99 random negatives produces mostly easy discrimination tasks. If Focal Loss's advantage is in hard cases, this evaluation may not detect it.

\subsection*{Recommended Change}
Add supplementary evaluation:

\begin{verbatim}
\textbf{Supplementary Evaluation (Hard Negatives):}
In addition to the standard 99-random-negative protocol, we
conduct a hard-negative evaluation:
\begin{itemize}
    \item For each test user, select 99 negatives from items
          that are popular but not interacted with by the user
    \item Alternatively, select negatives that the model
          (trained with BCE) ranks highest among all negatives
\end{itemize}
This protocol specifically tests whether Focal Loss improves
discrimination in challenging cases.
\end{verbatim}

%-----------------------------------------------------------------------
\section{Recommended Experimental Flow (Revised)}

\begin{enumerate}
    \item \textbf{Phase 1: Baseline Establishment}
    \begin{itemize}
        \item Train NeuMF-BCE with 1:10 negative sampling (primary)
        \item Train NeuMF-$\alpha$BCE with tuned $\alpha$, 1:10 sampling
        \item Tune hyperparameters on validation set
    \end{itemize}

    \item \textbf{Phase 2: Focal Loss Evaluation}
    \begin{itemize}
        \item Train NeuMF-FL with grid search over $\gamma$, $\alpha$
        \item Use same 1:10 sampling ratio
        \item Compare against both BCE and $\alpha$-BCE baselines
    \end{itemize}

    \item \textbf{Phase 3: Mechanism Validation}
    \begin{itemize}
        \item Run training dynamics analysis
        \item Verify that Focal Loss shifts gradient mass as theorized
    \end{itemize}

    \item \textbf{Phase 4: Robustness Study}
    \begin{itemize}
        \item Test across sampling ratios: 1:4, 1:10, 1:20, 1:50
        \item Analyze $\alpha$-sampling interaction
        \item Test hypothesis H2 (performance gap increases with imbalance)
    \end{itemize}

    \item \textbf{Phase 5: Evaluation Protocol Comparison}
    \begin{itemize}
        \item Standard 99-random evaluation
        \item Hard-negative evaluation
        \item Full-ranking on user subset (if computationally feasible)
    \end{itemize}
\end{enumerate}

%-----------------------------------------------------------------------
\section{Summary of Line-Referenced Edits}

\begin{tabular}{|l|p{8cm}|}
\hline
\textbf{Line(s)} & \textbf{Action} \\
\hline
421--428 & Replace hypothesis section with H1, H2, H3 \\
455 & Change primary sampling ratio from 4:1 to 10:1 \\
498--515 & Add $\alpha$-BCE row to Table 3 \\
554 & Add $\alpha$ for $\alpha$-BCE to Table 5 \\
580 & Add hard-negative evaluation description \\
598 & Insert Training Dynamics Analysis subsection \\
613 & Fix H7 reference (define or remove) \\
618--645 & Update ablation tables with new configurations \\
\hline
\end{tabular}

\end{document}
