%%%%%%%%%%%%%%%%%%%%%%%%%%%%%%%%%%%%%%%%%%%%%%%%%%%%%%%%%%%%%%%%%%%%%%%%

%%% LaTeX Template for AAMAS-2025 (based on sample-sigconf.tex)
%%% Prepared by the AAMAS-2025 Program Chairs based on the version from AAMAS-2025. 

%%%%%%%%%%%%%%%%%%%%%%%%%%%%%%%%%%%%%%%%%%%%%%%%%%%%%%%%%%%%%%%%%%%%%%%%

%%% Start your document with the \documentclass command.


%%% == IMPORTANT ==
%%% Use the first variant below for the final paper (including auithor information).
%%% Use the second variant below to anonymize your submission (no authoir information shown).
%%% For further information on anonymity and double-blind reviewing, 
%%% please consult the call for paper information
%%% https://aamas2025.org/index.php/conference/calls/submission-instructions-main-technical-track/

%%%% For anonymized submission, use this
%\documentclass[sigconf,anonymous]{aamas} 

%%%% For camera-ready, use this
\documentclass[sigconf]{aamas} 


%%% Load required packages here (note that many are included already).

\usepackage{balance} % for balancing columns on the final page


%%%%%%%%%%%%%%%%%%%%%%%%%%%%%%%%%%%%%%%%%%%%%%%%%%%%%%%%%%%%%%%%%%%%%%%%

%%% AAMAS-2025 copyright block (do not change!)

\setcopyright{ifaamas}
\acmConference[AAMAS '25]{Proc.\@ of the 24th International Conference
on Autonomous Agents and Multiagent Systems (AAMAS 2025)}{May 19 -- 23, 2025}
{Detroit, Michigan, USA}{A.~El~Fallah~Seghrouchni, Y.~Vorobeychik, S.~Das, A.~Nowe (eds.)}
\copyrightyear{2025}
\acmYear{2025}
\acmDOI{}
\acmPrice{}
\acmISBN{}


%%%%%%%%%%%%%%%%%%%%%%%%%%%%%%%%%%%%%%%%%%%%%%%%%%%%%%%%%%%%%%%%%%%%%%%%

%%% == IMPORTANT ==
%%% Use this command to specify your submission number.
%%% In anonymous mode, it will be printed on the first page.

\acmSubmissionID{7}

%%% Use this command to specify the title of your paper.

\title[Formatting Instructions]{A Comparative Analysis of Neural Collaborative Filtering and Standard Recommendation Systems using Focal Loss
}

%%% Provide names, affiliations, and email addresses for all authors.

\author{Rotem Even Zur, 208839183}
\affiliation{
  \institution{Ben Gurion University of the Negev}
  \city{Beer Sheva}
  \country{Israel}}
\email{evenzro@post.bgu.ac.il}

\author{Guy Kalati, 318366150}
\affiliation{
  \institution{Ben Gurion University of the Negev}
  \city{Beer Sheva}
  \country{Israel}}
\email{guykalat@post.bgu.ac.il}

\author{Dvir Chitrit, 206766818}
\affiliation{
  \institution{Ben Gurion University of the Negev}
  \city{Beer Sheva}
  \country{Israel}}
\email{dvirchi@post.bgu.ac.il}

\author{Omer Eliyahu, 209510828}
\affiliation{
  \institution{Ben Gurion University of the Negev}
  \city{Beer Sheva}
  \country{Israel}}
\email{omereliy@post.bgu.ac.il}

%%% Use this environment to specify a short abstract for your paper.

\begin{abstract}
%This document outlines the formatting instructions for submissions to
%AAMAS-2025. You can use its source file as a template when writing 
%your own paper. It is based on the file `\texttt{sample-sigconf.tex}'
%distributed with the ACM article template for \LaTeX\@.
\end{abstract}

%%% The code below was generated by the tool at http://dl.acm.org/ccs.cfm.
%%% Please replace this example with code appropriate for your own paper.


%%% Use this command to specify a few keywords describing your work.
%%% Keywords should be separated by commas.

%\keywords{Legends, Myths, Folktales}

%%%%%%%%%%%%%%%%%%%%%%%%%%%%%%%%%%%%%%%%%%%%%%%%%%%%%%%%%%%%%%%%%%%%%%%%

%%% Include any author-defined commands here.
         
\newcommand{\BibTeX}{\rm B\kern-.05em{\sc i\kern-.025em b}\kern-.08em\TeX}

%%%%%%%%%%%%%%%%%%%%%%%%%%%%%%%%%%%%%%%%%%%%%%%%%%%%%%%%%%%%%%%%%%%%%%%%

\begin{document}

%%% The following commands remove the headers in your paper. For final 
%%% papers, these will be inserted during the pagination process.

\pagestyle{fancy}
\fancyhead{}

%%% The next command prints the information defined in the preamble.

\maketitle 

%%%%%%%%%%%%%%%%%%%%%%%%%%%%%%%%%%%%%%%%%%%%%%%%%%%%%%%%%%%%%%%%%%%%%%%%

\section{Introduction}

%This document explains the main features of the `\texttt{aamas}' 
%document class, which is essentially identical to the `\texttt{acmart}'
%document class provided by the ACM. The only difference is a minor 
%modification to allow for the correct copyright attribution to IFAAMAS.
%For detailed documentation of the original document class, please refer
%to the relevant website maintained by the~ACM:
%
%\begin{center}
%\url{https://www.acm.org/publications/proceed%ings-template}
%\end{center}
%
%The first command in your source file should be either one of these:
%\begin{verbatim}
 %   \documentclass[sigconf,anonymous]{aamas}
 %   \documentclass[sigconf]{aamas}
%\end{verbatim}
%
%The first variant should be
%used when you submit your paper for blind review; it will replace the names of the authors with the submission number.
%The second variant should be used for final papers. 

%Make sure your paper includes the correct copyright information and 
%the correct specification of the \emph{ACM Reference Format}. Both of 
%these will be generated automatically if you include the correct 
%\emph{copyright block} as shown in the source file of this document.

%Modifying the template---e.g., by changing margins, typeface sizes, 
%line spacing, paragraph or list definitions---or making excessive use 
%of the `\verb|\vspace|' command to manually adjust the vertical spacing 
%between elements of your work is not allowed. You risk getting your 
%submission rejected (or your final paper excluded from the proceedings) 
%in case such modifications are discovered. The `\texttt{aamas}' document 
%class requires the use of the \textit{Libertine} typeface family, which 
%should be included with your \LaTeX\ installation. Please do not use 
%other typefaces instead.

%Please consult the \emph{Call for Papers} for information on matters 
%such as the page limit or anonymity requirements. It is available from
%the conference website:
%
%\begin{center}
%\url{https://aamas2025.org/}
%\end{center}
%
%To balance the columns on the final page of your paper, use the 
%`\texttt{balance}' package and issue the 
%`\verb|\balance|' command
 %somewhere in the text of what would be the %first column of the last 
 %page without balanced columns. This will be required for final papers.

%%%%%%%%%%%%%%%%%%%%%%%%%%%%%%%%%%%%%%%%%%%%%%%%%%%%%%%%%%%%%%%%%%%%%%%%

\section{Background}



%%%%%%%%%%%%%%%%%%%%%%%%%%%%%%%%%%%%%%%%%%%%%%%%%%%%%%%%%%%%%%%%%%%%%%%%

\section{Related Work}



%%%%%%%%%%%%%%%%%%%%%%%%%%%%%%%%%%%%%%%%%%%%%%%%%%%%%%%%%%%%%%%%%%%%%%%%

\section{Citations and References}
  


%%%%%%%%%%%%%%%%%%%%%%%%%%%%%%%%%%%%%%%%%%%%%%%%%%%%%%%%%%%%%%%%%%%%%%%%

%%% The acknowledgments section is defined using the "acks" environment
%%% (rather than an unnumbered section). The use of this environment 
%%% ensures the proper identification of the section in the article 
%%% metadata as well as the consistent spelling of the heading.

%\begin{acks}
%If you wish to include any acknowledgments in your paper (e.g., to 
%people or funding agencies), please do so using the `\texttt{acks}' 
%environment. Note that the text of your acknowledgments will be omitted
%if you compile your document with the `\texttt{anonymous}' option.
%\end{acks}

%%%%%%%%%%%%%%%%%%%%%%%%%%%%%%%%%%%%%%%%%%%%%%%%%%%%%%%%%%%%%%%%%%%%%%%%

%%% The next two lines define, first, the bibliography style to be 
%%% applied, and, second, the bibliography file to be used.

\bibliographystyle{ACM-Reference-Format} 
\bibliography{sample}

%%%%%%%%%%%%%%%%%%%%%%%%%%%%%%%%%%%%%%%%%%%%%%%%%%%%%%%%%%%%%%%%%%%%%%%%

\end{document}

%%%%%%%%%%%%%%%%%%%%%%%%%%%%%%%%%%%%%%%%%%%%%%%%%%%%%%%%%%%%%%%%%%%%%%%%

